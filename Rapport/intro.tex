\chapter{Introduction}

\section{Contexte}

Posséder un outil permettant l'automatisation du traitement des données de facturation est un besoin présent non seulement en entreprise mais également dans le cadre d'une utilisation personnelle.\\

L'élément clé permettant d'automatiser le traitement de factures est le concept de reconnaissance optique de caractères, ou OCR. 
Les premiers systèmes d'OCR ont été développés durant les années 1970 et longtemps maintenus sous secret professionnel. Des dispositifs d'OCR open-source ont finalement fait leur apparition en ligne début 2000.\\

Le nombre d'applications proposant l'informatisation de factures est aujourd'hui impressionnant; cependant aucun de ces logiciels ne semble se démarquer totalement.\\

On peut facilement imaginer le potentiel d'une telle application si, en parallèle de sa fonction première, elle classe et interprète les données traitées en les conservant de manière anonyme.\\

C'est dans cet esprit que nous avons développé notre propre application de traitement de factures à partir d'images, en nous appuyant sur la librairie OCR open-source Tessaract.

\section{Organisation du document}
Ce document regroupe les différentes spécifications du projet, présentées principalement à travers le langage UML selon les méthodes apprises en cours de GL52 ce semestre. Le système est donc étudié dans un premier temps de façon globale : la granularité de l'analyse s'affine ensuite au cours du développement du projet.\\
Dans un premier temps, des use case globaux et leurs diagrammes de séquences associés présentent l'interface utilisateur et les traitement nécessaires du côté système. Par la suite, l'affinement des besoins apporte des précisions quant au découpage des différentes fonctions et classes en packages. Enfin, ces derniers sont définis précisément via un diagramme de classes détaillé et des diagrammes de séquence raffiné.

\section{Définitions et abréviations}
\begin{enumerate}[label=\bfseries]
	\item[{OCR :}] Optical Caracter Recognition \\
	Traduction d'une image de texte imprimé ou manuscrit en fichier de texte.
\end{enumerate}
